% 
%            ,,                                        
%          `7MM            _.o9                                
%            MM                                             
%  ,6"Yb.    MM  ,p6"bo   ,6"Yb.  M"""MMV  ,6"Yb.  `7Mb,od8 
% 8)   MM    MM 6M'  OO  8)   MM  '  AMV  8)   MM    MM' "' 
%  ,pm9MM    MM 8M        ,pm9MM    AMV    ,pm9MM    MM     
% 8M   MM    MM YM.    , 8M   MM   AMV  , 8M   MM    MM     
% `Moo9^Yo..JMML.YMbmd'  `Moo9^Yo.AMMmmmM `Moo9^Yo..JMML.   
% 
% 
% Free and Open-Source template for academic works
% https://github.com/dpmj/alcazar



\clearpage
\cleardoublepage

\chapter{Introduction}

\section{Alcázar}

This is an example of how this template may be used. As we can see in \autoref{fig:cha1:donpedro} \cite{cubesat-impact-astronomy, lora-phy-understanding}, the glossary entry \glsname{antenna} is lorem ipsum dolor sit amet  \cite{nasa-soa2023-avionics, book-product-devel, maral-satcoms, solder-defects}.

\lipsum[12] \cite{cubesat-impact-astronomy}.


\lipsum[1] \cite{solder-defects}.

\section{Scelerisque id lorem in}


\lipsum[13]. Condimentum ipsum \glsname{emc}/\glsname{emi} and \glsname{esd} \cite{grounding-schemes-sats}. 
\begin{enumerate}
    \item Nulla tristique mi eget semper luctus. Etiam commodo vestibulum vulputate. Etiam quis sapien dolor. Nunc tristique eu lacus quis ullamcorper. 
    \item Sed volutpat rutrum vehicula. Donec nunc nisl, suscipit in faucibus vitae, tristique eu risus. Nulla facilisis augue eget interdum rutrum. Aliquam sem nunc, fermentum sed urna ac, faucibus interdum nisi.
\end{enumerate}
\begin{equation}
        a_1^2 + b_1^2 = c_1^2
\end{equation}

\newpage

\subsection{Dolor sit amet}

\lipsum[14-16]

\begin{figure}
    \centering
    \includegraphics[width=\linewidth]{figures/examples/example3.jpg}
    \caption[Fachada del Palacio del rey don Pedro, en el Real Alcázar de Sevilla.]{Fachada del Palacio del rey don Pedro, en el Real Alcázar de Sevilla. \href{https://commons.wikimedia.org/wiki/File:Fachada_del_Palacio_del_rey_don_Pedro.jpg}{Alberto Bravo}.}
    \label{fig:cha1:donpedro}
\end{figure}

\lipsum[17]

\subsubsection{Nullam quis lacus}

\lipsum[18-20] \cite{lora-hw-e539v03a}. 

\subsubsection{Nulla elementum orci}

\lipsum[21]

\begin{itemize}
    \item Proin a condimentum nibh. Praesent vulputate tellus vel metus rutrum, non luctus mi sollicitudin. 
    \begin{itemize}
        \item Nam ac tellus ut eros sollicitudin luctus at ac mi. Vestibulum mollis nec nisi a laoreet. Proin neque tortor, placerat nec suscipit sit amet, ullamcorper in sem.
        \item Fusce faucibus ultrices cursus. Maecenas scelerisque mauris diam, at volutpat nisi porta vitae. 
        \item Sed at ipsum et leo cursus varius eu eu lectus. 
        \item Class aptent taciti sociosqu ad litora torquent per conubia nostra, per inceptos himenaeos. Ut felis ipsum, imperdiet rhoncus orci ac, consectetur luctus nisl. 
    \end{itemize}
    \item Cras aliquet elementum tellus ullamcorper malesuada. Integer purus est, pharetra eu ullamcorper quis, imperdiet non turpis.
    \item In vestibulum faucibus ligula eget blandit. 
\end{itemize}

\lipsum[22], \autoref{code:cha:1:opening} \lipsum[23]
\texttt{Example monospace text.}

\begin{code}
    \captionof{listing}{Some example code}
    \label{code:cha:1:opening}
    \inputminted[]{tex}{main.tex}
\end{code}

\subsubsection{Nulla elementum orci}

\paragraph{Nullam quis lacus vel ante feugiat efficitur id ut quam}

\lipsum[23]

\begin{quotation}
    No es mas limpio el que más limpia,
\end{quotation}

\paragraph{Lorem ipsum dolor sit amet}

\lipsum[24-25]

\begin{quotation}
    Sino el que menos ensucia.
\end{quotation}

\subparagraph{Proin a condimentum nibh. Praesent vulputate tellus vel metus rutrum}

\lipsum[26]

\begin{quote}
    ¿Seguro? No sé yo...
\end{quote}

\lipsum[27]

\section{In vestibulum faucibus}

\lipsum[28-30]

\begin{figure}
    \centering
    \subfloat[Palacio de Villavicencio, Alcázar, Jerez de la Frontera.][Palacio de Villavicencio, Alcázar, Jerez de la Frontera. \href{https://es.wikipedia.org/wiki/Archivo:Mezquita,_Alc\%C3\%A1zar,_Jerez_de_la_Frontera,_Espa\%C3\%B1a,_2015-12-07,_DD_58.JPG}{Diego Delso}.]{
        \includegraphics[width=0.45\linewidth]{figures/examples/example4.jpg}
            \label{fig:cha1:jerez:villavicencio}
    }
    \hfill
    \subfloat[Mezquita, Alcázar, Jerez de la Frontera, España][Mezquita, Alcázar, Jerez de la Frontera, España \href{https://es.wikipedia.org/wiki/Archivo:Mezquita,_Alc\%C3\%A1zar,_Jerez_de_la_Frontera,_Espa\%C3\%B1a,_2015-12-07,_DD_58.JPG}{Diego Delso}.]{
        \includegraphics[width=0.45\linewidth]{figures/examples/example2}
            \label{fig:cha1:jerez:mezquita}
    }
    
    \caption{Alcázar de Jerez de la Frontera.}
    \label{fig:cha1:jerez}
\end{figure}

\lipsum[31-32]
% https://tex.stackexchange.com/questions/263809/matrix-equation
\begin{gather}
    \begin{bmatrix} \Phi_{11} & \Phi_{12} \\ \Phi_{21} & \Phi_{22} \end{bmatrix}
    =
    \frac{1}{\det(X)}
     \begin{bmatrix}
      X_{22} Y_{11} - X_{12} Y_{21} &
      X_{22} Y_{12} - X_{12} Y_{22} \\
      X_{11} Y_{21} - X_{21} Y_{11} &
      X_{11} Y_{22} - X_{21} Y_{12} 
      \end{bmatrix}
   \end{gather}

\lipsum[33-34]
% https://tex.stackexchange.com/questions/321796/how-to-handle-long-equation-in-latex
\begin{equation}
    \begin{split}
    f_{r:n}(x)&=\frac{n!}{(r-1)!\,(n-r+1)!}
       \Biggl[ \biggl( \frac{x-1}{k}  \biggr)^{\!r-1} 
               \biggl( \frac{k-x+1}{k}\biggr)^{\!n-r+1} \\
      &\qquad -\biggl( \frac{x}{k}    \biggr)^{\!r-1} 
               \biggl( \frac{k-x}{k}  \biggr)^{\!n-r+1} \,
       \Biggr]+ f_{r-1:n}(x)                    
    \end{split}
\end{equation}


\begin{landscape}
    \begin{figure}[p]
        \centering
        \includegraphics[width=0.69\linewidth]{figures/examples/example5.jpg}
        \caption[Plano. Sección A-A y Planta Baja del Patio de las Doncellas.]{Plano. Sección A-A y Planta Baja del Patio de las Doncellas. \href{https://hdl.handle.net/11532/333959}{Instituto Andaluz del Patrimonio Histórico.}}
        \label{fig:drawing-doncellas}
    \end{figure}
\end{landscape}

\lipsum[45-44]
