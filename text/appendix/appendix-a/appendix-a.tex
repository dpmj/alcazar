% 
%            ,,                                        
%          `7MM            _.o9                                
%            MM                                             
%  ,6"Yb.    MM  ,p6"bo   ,6"Yb.  M"""MMV  ,6"Yb.  `7Mb,od8 
% 8)   MM    MM 6M'  OO  8)   MM  '  AMV  8)   MM    MM' "' 
%  ,pm9MM    MM 8M        ,pm9MM    AMV    ,pm9MM    MM     
% 8M   MM    MM YM.    , 8M   MM   AMV  , 8M   MM    MM     
% `Moo9^Yo..JMML.YMbmd'  `Moo9^Yo.AMMmmmM `Moo9^Yo..JMML.   
% 
% 
% Free and Open-Source template for academic works
% https://github.com/dpmj/alcazar


% Example of an appendix


\chapter{Something vaguely related to the project.}


\section{But boy I worked so hard in this}


\lipsum[70] \cite{europa_rohs, europa_ce}.


\subsection{You are going to read it}

\lipsum[71-72] 


\subsection{Praesent vulputate tellus vel metus rutrum}

\lipsum[73-75]  \cite{ds-sx1276, ds-sx1302, an1368-analog-ferrite}

\begin{figure}
    \centering
    \includegraphics[width=\linewidth]{figures/examples/example1.jpg}
    \caption[Alcázar de los reyes cristianos, Córdoba.]{Alcázar de los reyes cristianos, Córdoba. \href{https://es.wikipedia.org/wiki/Archivo:Alcazar_Cordoba.jpg}{Ahura klik}.}
    \label{fig:apxA:cordoba}
\end{figure}

\lipsum[76] 

\begin{description}
    \item[Nullam quis lacus] vel ante feugiat efficitur id ut quam. Pellentesque commodo elit nec urna gravida maximus. Suspendisse ut risus eu ipsum porta porta ac et orci. Donec dictum ligula sodales, euismod est sed, semper libero \glsname{gnu}. 
    \item[In blandit], nulla et elementum pharetra, mi nunc sagittis tellus, sit amet scelerisque magna elit ac sapien. Curabitur ipsum dui, pretium a maximus id, varius gravida nisl. Sed vitae mattis elit, vitae hendrerit lorem \cite{an1368-analog-ferrite, itu-r-reg-articles}.
\end{description}

\subsubsection{Lorem ipsum dolor sit amet, consectetur adipiscing elit}

\lipsum[77] 

\subsubsection{Fusce faucibus ex at massa ultrices elementum}

\lipsum[78] 
\newpage

\section{Nulla elementum orci in dolor dapibus}

\lipsum[79] 

\subsection{Non luctus mi sollicitudin}

\lipsum[80-82] 

\newpage 

\begin{code}
\captionof{listing}{Some example large code}
\label{code:apx:a:python}
\begin{minted}{python}
"""
Some interesting python code 
Blah Blah Blah
"""

import numpy as np
from matplotlib import pyplot as plt
from numpy.polynomial import Polynomial as Poly

plt.rcParams["font.family"] = "Libertinus Serif"
plt.rcParams['font.size'] = 14
plt.rcParams['mathtext.fontset'] = 'custom'
plt.rcParams['mathtext.rm'] = 'Libertinus Serif'
plt.rcParams['mathtext.it'] = 'Libertinus Serif:italic'
plt.rcParams['mathtext.bf'] = 'Libertinus Serif:bold'

R510 = 508.3  # Ohms
V5 = 5.018  # Volts

# Open files

nmos_vgs, nmos_vr = np.loadtxt('nmos_ids_vgs.csv', delimiter='\t', unpack=True, skiprows=1)
nmos_sim_vgs, nmos_sim_ids = np.loadtxt('CIC_P0_NMOS_ids_vgs_5.txt', delimiter='\t', unpack=True, skiprows=1)

# Do thingies

nmos_ids = nmos_vr / R510  # Current in resistor
p = Poly.fit(nmos_vgs, nmos_ids, 1)  # Tendency line
print(f"f(x) = {p:unicode}")

# Plot stuff

fig0, ax0 = plt.subplots(figsize=[12, 6])
x = np.linspace(0, 6, 100)

plt.plot(nmos_vgs, nmos_ids, linestyle='none', marker=".", label='Experimental')
plt.plot(nmos_sim_vgs, nmos_sim_ids, linestyle='dashdot', label='Simulation')
plt.plot(x, p(x), linestyle='dashed', label='Tendency')

ax0.grid(True, which='major', color='#DDDDDD', linestyle='-', linewidth=0.6)
ax0.grid(True, which='minor', color='#DDDDDD', linestyle=':', linewidth=0.6)
ax0.minorticks_on()

ax0.legend()

plt.title(r'NMOS, curve $I_{DS}$ / $V_{GS}$')
plt.xlabel(r'$V_{GS}$ (V)')
plt.ylabel(r'$I_{DS}$ (A)')

plt.xlim([0, 6])

plt.savefig("nmos_vgs_ids.pdf")
\end{minted}
\end{code}

